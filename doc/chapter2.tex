\chapter{Governing Equations}
\label{math}

\section{Fluid Modeling}
In order to model detonations, the problem can be broken up into smaller pieces and solved with CFD. OpenFOAM is computational toolbox that will utilize the finite volume method (FVM) to solve the set of partial differential equations describing the fluid flow. In the FVM, the computational domain is broken into multiple smaller control volumes, known as cells. Each cell contains a set of thermo-fluid properties at the centroid. The differential equations are then integrated over this control volume in order to obtain a solution at each cell. The solution is interpolated between cell centroids to obtain a solution throughout the domain. 

The solution for the fluid problem explored in this thesis work is approximated through density, momentum, species, and energy equations in that order. The solver used to approximate these equations in a discretized computational domain, \verb|rhoReactingCentralFoam|, utilizes the central-upwind schemes developed by Kurganov and Tadmor \cite{kurganov1}. The density equation solved by \verb|rhoReactingCentralFoam| is
\begin{equation}
\frac{\partial \rho}{\partial t} + \nabla \cdot \left(\rho \bm{u}\right) = 0,
\end{equation}
the momentum equation is
\begin{equation}
\frac{\partial \rho\bm{u}}{\partial t} + \nabla \cdot \left(\rho \bm{u}\otimes \bm{u}\right) + \nabla p = \bm{0}, 
\end{equation}
the energy equation is
\begin{equation}
\frac{\partial \rho E}{\partial t} + \nabla \cdot \left[\left(\rho E + p\right)\bm{u}\right] = \dot{q}
\end{equation}
and the specie equation is
\begin{equation}
\frac{\partial \rho Y_i}{\partial t} + \nabla \cdot \left(\rho Y_i \bm{u}\right) = \dot{\omega}_i
\end{equation}
where 
\begin{equation}
\dot{q} = \sum_{i = 1}^N \dot{\omega_i} \Delta h_{f,i}^0
\end{equation}
and
\begin{tabbing}
\qquad \= \(\dot{\omega}_i\) \qquad \= specie source reaction rate \\ 
\> \(\dot{q}\) \> energy source \\
\> \(\rho\) \> density \\
\> \(\bm{u}\) \> fluid velocity vector \\
\> \(Y_i\) \> specie mass fraction \\
\> \(E\) \> total energy \\
\> \(p\) \> pressure\\
\> \(\Delta h_{f,i}^0\) \> specie formation enthalpy
\end{tabbing}

\noindent These sets of equations \cite{kuo}, known as the reactive Euler equations, are solved using OpenFOAM's differential equation time step integrators. Different equation solvers can be selected for different quantities of interest. For quantities like velocity, internal energy, and specie, the preconditioned biconjugate gradient stabilized method is used (\verb|PBiCGStab| in OpenFOAM). For solvers like this, preconditioners can be selected as well as solution tolerance and relative tolerance. Preconditioners are computational operations done on the solution matrix in order to help prepare them for numerical iterative solving such that less iterations are required. The system is closed when energy is expanded and an equation of state is added. Total energy \cite{kuo} can be written as 
\begin{equation}
E = h - \frac{p}{\rho} +\frac{1}{2} \left(\bm{u}\cdot\bm{u}\right)
\end{equation}
where enthalpy is written \cite{kuo} as 
\begin{equation}
h = \sum_{i = 1}^Nh_{s,i}Y_i
\end{equation}
and specie total enthalpy \cite{kuo} as 
\begin{equation}
h_{i} = \Delta h_{f,i}^0 + h_{s,i}
\end{equation}
with the sensible enthalpy 
\begin{equation}
h_{s,i} = \int_{T_0}^T C_{p,i}\mathrm{d}T
\end{equation}
where 
\begin{tabbing}
\qquad \= \(h\) \qquad \= total summed enthalpy\\ 
\> \(C_{p,i}\) \> specie specific heat\\
\> \(h_i\) \> specie total enthalpy \\
\> \(h_{s,i}\) \> specie sensible enthalpy \\
\> \(T\) \> reacting temperature \\
\> \(T_0\) \> initial temperature 
\end{tabbing}
The equation of state is expressed as 
\begin{equation}
p = \rho R T
\end{equation}
Within OpenFOAM, a user has choices as to how he or she would like to numerically solve the stiff set of differential equations that model reacting flows. Thermodynamic variables such as specific heat can be selected to be temperature-dependent using polynomial approximations such as those given by the JANAF tables, such that \(C_{p,i} = C_{p,i}(T)\). No significant complexity in solution was noticed, so JANAF settings were used for temperature-dependent thermodynamics. The energy equation solution variable can also be selected, such that either sensible enthalpy or sensible internal energy is solved, and we selected sensible internal energy for modeling detonations with \verb|rhoReactingCentralFoam| as less noise in the solution was noticed over sensible enthalpy. Transport modeling can also be changed, and for this research the Sutherland model was used \cite{ofug}:
\begin{equation}
\mu = \frac{A_s \sqrt{T_s}}{1 + \frac{T_s}{T}} 
\end{equation}
where
\begin{tabbing}
\qquad \= \(\mu\) \qquad \= dynamic viscosity\\ 
\> \(A_s\) \> Sutherland coefficient \\
\> \(T_s\) \> Sutherland temperature\\
\end{tabbing}
The numerical solution of the Euler equations in OpenFOAM occurs with several different settings. Different numerical schemes can be applied separately to variables of interest, and can be seen in Table \ref{tab:numschemes}. 

\begin{table}[h]
\centering
\caption{OpenFOAM finite volume numerical schemes applied during solving}
\label{tab:numschemes}
\begin{tabular}{ccc}
Term & OpenFOAM Variable & Numerical Scheme \\ \hline 
Flux Scheme & \verb|fluxScheme| & Kurganov \\ 
Time Scheme & \verb|ddtSchemes| & Euler \\
Gradient Schemes & \verb|gradSchemes| & Gauss linear \\ 
Divergence Schemes & \verb|divSchemes| & none by default \\ 
& \verb|div(tauMC)| & linear \\ 
& \verb|div(phi,|specie\verb|)| & van Leer \\ 
Laplacian Schemes & \verb|laplacianSchemes| & Gauss linear uncorrected \\ 
Interpolation Schemes & \verb|interpolationSchemes| & default linear \\
& \verb|reconstruct(rho)| & Minmod \\ 
& \verb|reconstruct(U)| & MinmodV \\ 
& \verb|reconstruct(T)| & Minmod \\ 
& \verb|reconstruct(Yi)| & Minmod \\ 
Surface Normal Gradient Schemes & \verb|snGradSchemes| & uncorrected \\
\end{tabular}
\end{table}
In Table \ref{tab:numschemes}, Gauss linear specifies two things \cite{ofug}. The ``Gauss'' portion tells us that OpenFOAM should use a standard finite volume Gaussian integration with interpolation of values of interest between face centers and cell centers. The ``linear'' portion tells OpenFOAM that piecewise linear interpolation should be used for the interpolation. It should be noted that the ``uncorrected'' term seen is a flag that tells OpenFOAM that the orthogonal component of a flow variable with respect to a cell face should not undergo a non-orthogonal correction, since structured grids are used in this research. ``Corrected'' is more useful for engineering geometries where the mesh faces are not always aligned with a Cartesian coordinate system. The next important scheme used is van Leer. van Leer \cite{vanleer} is a flux limiting scheme that can limit the solution gradient near regions like shocks. Cell interpolation with this method is piecewise linear, but the slopes are limited. van Leer requires using the linear Gudunov scheme that assumes a discontinuous solution on ever finite volume cell face. Like van Leer, Minmod is a slope limiter, but uses a different set of criteria for the limiting. Time schemes are done with the Euler scheme, a transient and first order implicit scheme. This scheme relies on knowledge of the derivative and therefore both the current and subsequent solution in order to give an approximation of the solution at the next time step. By knowing the slope of the solution, we can guess the solution at the next time step by tracing out where it should be by this rate of change. The last numerical scheme used is the Kurganov scheme \cite{kurganov1}, a central difference scheme by Kurganov and Tadmor that is second order and high resolution. 

Once the numerical schemes are defined, the system of equations needs to be solved. The mass, momentum, energy, and specie equations are combined together in matrix linear form to be solved (\(\bm{Ax} = \bm{b}\)). OpenFOAM allows the user to alter the solver settings for each solution variable\cite{ofug}. The velocity, internal energy, and specie mass fraction utilize the PBiCGStab solver mentioned earlier. This is an iterative linear solver that uses a preconditioner to help accelerate convergence of the solution.  Computationally, there are two matrix-vector multiplications, six parallel reductions, and a twice-applied preconditioning step that occurs every time step. Upon startup there is an initial matrix-vector multiplication and a single parallel reduction. The preconditioner used for enthalpy and specie is known as DILU, or Diagonal-based Incomplete LU. This preconditioner performs an incomplete LU factorization of the solution matrix and reduces the solver's sensitivity to errors or changes in input. The preconditioner for velocity is DIC, or Diagonal-based Incomplete Cholesky. This preconditioner performs an incomplete Cholesky factorization of the solution matrix to precondition the solution matrix. Density is solved using the \verb|diagonal| solver, a simple solve by a solution matrix inversion. Parameters used for numerical solving can be seen in Table \ref{tab:numerics}. 

\begin{table}[h]
\centering
\caption{Finite volume numerical solvers and preconditioners used for solution variables}
\label{tab:numerics}
\begin{tabular}{cccc}
Variable & Solver & Parameter & Value \\ \hline 
\verb|e|, \verb|Yi| & \verb|PBiCGStab| & Preconditioner & \verb|DILU| \\ 
& & Tolerance & \verb|1e-17| \\ 
& & Relative tolerance & 0 \\
\verb|U| & \verb|PBiCGStab| & Preconditioner & \verb|DIC| \\ 
& & Tolerance & \verb|1e-15| \\ 
& & Relative tolerance & 0 \\
\verb|rho| & \verb|diagonal| & & \\
\end{tabular}
\end{table}


\section{Reaction Modeling and Chemistry}
Since detonation modeling consists of both a fluids problem as well as chemical problem, chemical reactions must also be considered. Like a fire, a detonation requires a fuel and an oxidizer. Typically detonations in experimental and computational modeling will utilize gaseous hydrogen or methane fuels mixed with air or pure oxygen oxidizers. For the computational modeling here, hydrogen-air was the primary respective fuel-oxidizer combination utilized for exploring detonation modeling. This was due to the highly reactive nature of hydrogen as well as previous computational and experimental modeling resources available. The reaction used \cite{kuo} for the hydrogen detonation modeling here is
\begin{center}
%\ch{Na2SO4 ->[ H2O ] Na+ + SO4^2-}
\ch{2 H2 + 2 (O2 + 3.76 N2) -> 2 H2O + 7.52 N2}
\end{center}
This is a stoichiometric reaction between diatomic hydrogen gas and air. When converted to mass fraction \(Y\) for OpenFOAM \cite{marcantoni}, it becomes Table \ref{tab:y}.
\begin{table}[H]
\centering
\caption{Specie mass fraction initial condition}
\label{tab:y}
\begin{tabular}{cc}
Specie & Mass Fraction \\ \hline
H\(_2\) & 0.02851 \\ 
H\(_2\)O & 0 \\
N\(_2\) & 0.745 \\ 
O\(_2\) & 0.226 \\ 
Total & 0.99951 \\ 
\end{tabular}
\end{table}

Other trace elements present in air are not modeled here, and this could be a topic of further study within the context of OpenFOAM detonation modeling with AMR. Additionally, a single-step reaction was modeled as opposed to several reactions together to model hydrogen-air detonations. While more steps are certainly more realistic, adding further reaction complexity was found to significantly increase computational expense. It was decided that gaining a better baseline understanding of how AMR could improve modeling efficiency in OpenFOAM was the focus here and thus further work on increasing the reaction complexity and accuracy was not performed. 

The chemical reaction rate can be modeled OpenFOAM with the Arrhenius equation \cite{christ} which takes the form: 
\begin{equation}
\dot{\omega}_i = AT^\beta \exp\left(\frac{Ea}{R T}\right)
\end{equation}
where 
\begin{tabbing}
\qquad \= \(\dot{\omega}_i\) \qquad \= specie source reaction rate \\ 
\> \(A\) \> pre-exponential factor \\
\> \(T\) \> temperature \\
\> \(\beta\) \> temperature exponent \\
\> \(Ea\) \> activation energy \\
\> \(R\) \> specific gas constant 
\end{tabbing}

\noindent OpenFOAM requests \verb|A|, \verb|beta|, and \verb|Ta|. We can write \(Ta = \frac{Ea}{R }\). While in later sections the effect of \(A\) is explored on the solution, the values landed on for numerical simulation in this work are:
\begin{equation}
   A = 1.4 \times 10^{13} ~ \text{m}^3\text{mol}^{-1}s^{-1},
   \qquad 
   Ta = 12996 ~\text{K},
   \qquad
   \beta = 0
\end{equation}
\noindent In addition, the specific gas constant used is \(R = 368.9\) J/Kg-K. It is seen in Chapter \ref{sec:a} that the values used are reasonably close to Chapman-Jouguet theory as well as other published values \cite{towery1}\cite{hashemi}. 

The chemistry itself uses a separate solver than the rest of the reacting flow solution variables. The settings for the chemistry solver are found in \verb|constant/chemistryProperties|. The directory structure of OpenFOAM is discussed in Chapter \ref{ofchap} in greater detail. The chemistry is solved before the flow variables, with a separate initial time step. For detonation modeling presented here, the chemistry ordinary differential equation (ODE) is solved using the Runge-Kutta-Fehlberg method \cite{rkf}, known to OpenFOAM as RKF45. 

Together, the numerical schemes, numerical solvers, and chemistry solver combine to solve the reactive Euler equations as \verb|rhoReactingCentralFoam| within OpenFOAM. These solution techniques can then be applied to flow problems such as detonations. 

