\section{AMR-Based Detonation Solver in OpenFOAM}

\begin{frame}{OpenFOAM Overview}
\begin{itemize}
\item Open-source, free, computational fluid dynamics toolbox 
\item Written in C++, has many solvers 
\item No GUI, file-based input and control
\end{itemize}
\end{frame}

\begin{frame}{Tested Solvers}
Solvers tested for their capability to model shocks and detonations:
\begin{itemize}
\item \textbf{rhoReactingFoam}: included with OpenFOAM, a density-based combustion solver
\item \textbf{rhoCentralFoam}: included with OpenFOAM and developed by Greenshields \textit{et. al.} \cite{greenshields}, a density-based solver that uses the central-upwind schemes of Kurganov and Tadmor \cite{kurganov1} 
\item \textbf{rhoReactingCentralFoam}: a solver combined by Caelan Lapointe with previous work done by Nakul \cite{nakul}, with AMR support
\end{itemize}
\end{frame}

\begin{frame}[allowframebreaks]{Governing Equations}
Detonations were modeled using the reacting Navier-Stokes equations \cite{kuo,stokes}:
\begin{equation}
\frac{\partial \rho}{\partial t} + \nabla \cdot \left(\rho \bm{u}\right) = 0\,
\end{equation}
\begin{equation}
\frac{\partial \rho\bm{u}}{\partial t} + \nabla \cdot \left(\rho \bm{u}\otimes \bm{u}\right) + \nabla p -\mu\nabla^2\bm{u}= \bm{0}\,,
\end{equation}
\begin{equation}
\frac{\partial \rho E}{\partial t} + \nabla \cdot \left[\left(\rho E + p\right)\bm{u}\right] -\alpha\nabla^2 e = \dot{q}\,,
\end{equation}
\begin{equation}
\frac{\partial \rho Y_i}{\partial t} + \nabla \cdot \left(\rho Y_i \bm{u}\right) -\mu\nabla^2 Y_i= \dot{\omega}_i\,,
\end{equation}
where 
\begin{equation}
\dot{q} = \sum_{i = 1}^N \dot{\omega_i} \Delta h_{f,i}^0\,,
\end{equation}
%is the heat flux, $\dot{\omega}_i$ is the species source reaction rate, $\rho$ is the density, $\bm{u}$ is the fluid velocity vector, $Y_i$ is the mass fraction of the $i$th species, $E$ is the total energy, $p$ is the pressure, $\mu$ is the dynamic viscosity, $e$ is the internal energy, $\alpha$ is the thermal diffusivity and $\Delta h_{f,i}^0$ is the species formation enthalpy.

Total energy \cite{kuo} can be written as 
\begin{equation}
E = h - \frac{p}{\rho} +\frac{1}{2} \left(\bm{u}\cdot\bm{u}\right)\,,
\end{equation}
where the total summed enthalpy is written \cite{kuo} as 
\begin{equation}
h = \sum_{i = 1}^Nh_{s,i}Y_i\,,
\end{equation}
and the species total enthalpy \cite{kuo} is given by 
\begin{equation}
h_{i} = \Delta h_{f,i}^0 + h_{s,i}\,,
\end{equation}
with the sensible enthalpy for the $i$th species expressed as
\begin{equation}
h_{s,i} = \int_{T_0}^T C_{p,i}\mathrm{d}T\,,
\end{equation}
Here $C_{p,i}$ is the specific heat for the $i$th species, $T$ is the temperature, and $T_0$ is an initial, or reference, temperature. The equation of state is expressed as 
\begin{equation}
p = \rho R T\,,
\end{equation}
where $R=R_u/W$. Specific heat $C_{p,i}=C_{p,i}(T)$ from NIST JANAF \cite{janaf} lookup tables.  
\end{frame}

\begin{frame}{Notes About Governing Equations}
\begin{itemize}
\item No turbulence modeling 
\begin{itemize}
    \item subgrid-scale turbulence structures averaged out numerically
    \item akin to implicit LES
\end{itemize}
\item Not modeling inviscid Euler equations; viscosity is accounted for with Sutherland \cite{sutherland} model:
\end{itemize}
\begin{equation}
\mu = \frac{A_s \sqrt{T_s}}{1 + \frac{T_s}{T}} \,,
\end{equation}
\end{frame}

\begin{frame}{Chemical Reactions}
Stoichiometric hydrogen-air utilized here, which follows the following expression \cite{kuo}:
\begin{center}
\ch{2 H2 + 2 (O2 + 3.76 N2) -> 2 H2O + 7.52 N2}
\end{center}
with
\begin{table}[t!]
\centering
\begin{tabular}{cc}
Species & Mass Fraction \\ \hline
H\(_2\) & 0.02851 \\ 
H\(_2\)O & 0 \\
N\(_2\) & 0.745 \\ 
O\(_2\) & 0.226 \\ \hline
Total & 0.99951 \\ 
\end{tabular}
\end{table}
\end{frame}

\begin{frame}{Reaction Rate Modeling}
Arrhenius equation \cite{arrhenius} takes the form \cite{christ} 
\begin{equation}
\dot{\omega}_i = AT^\beta \exp\left(\frac{E_a}{R T}\right)\,,
\end{equation}
where $Ta = Ea/R$. Simulation values were explored, but we settled on 
\begin{equation}
   A = 1.4 \times 10^{13} ~ \text{m}^3\text{mol}^{-1}\text{s}^{-1},
   \qquad 
   Ta = 12996 ~\text{K},
   \qquad
   \beta = 0\,.
\end{equation}
with \(R = 368.9\) J/Kg-K. As shown later these reasonably match Chapman-Jouguet detonation theory \cite{chapman} along with other published values \cite{towery1,hashemi}.
\end{frame}


\begin{frame}{OpenFOAM Finite Volume Numerical Schemes}
\begin{table}[t!]
\centering
%\caption{OpenFOAM finite volume numerical schemes applied during solving}
%\label{tab:numschemes}
\scalebox{0.9}{
\begin{tabular}{ccc}
Term & OpenFOAM Variable & Numerical Scheme \\ \hline 
Flux Scheme & \texttt{fluxScheme} & Kurganov \\ 
Time Scheme & \texttt{ddtSchemes} & Euler \\
Gradient Schemes & \texttt{gradSchemes} & Gauss linear \\ 
Divergence Schemes & \texttt{divSchemes} & none by default \\ 
& \texttt{div(tauMC)} & linear \\ 
& \texttt{div(phi,}specie\texttt{)} & van Leer \\ 
Laplacian Schemes & \texttt{laplacianSchemes} & Gauss linear uncorrected \\ 
Interpolation Schemes & \texttt{interpolationSchemes} & default linear \\
& \texttt{reconstruct(rho)} & Minmod \\ 
& \texttt{reconstruct(U)} & MinmodV \\ 
& \texttt{reconstruct(T)} & Minmod \\ 
& \texttt{reconstruct(Yi)} & Minmod \\ 
Surface Normal Gradient Schemes & \texttt{snGradSchemes} & uncorrected \\
\end{tabular}
}
\end{table}
\end{frame}

\begin{frame}{OpenFOAM Finite Volume Numerical Solvers}
\begin{table}[t!]
\centering
%\caption{Finite volume numerical solvers and preconditioners used for solution variables}
%\label{tab:numerics}
\begin{tabular}{cccc}
Variable & Solver & Parameter & Value \\ \hline 
\texttt{e}, \texttt{Y} & \texttt{PBiCGStab} & Preconditioner & \texttt{DILU} \\ 
& & Tolerance & \texttt{1e-17} \\ 
& & Relative tolerance & 0 \\\hline
\texttt{U} & \texttt{PBiCGStab} & Preconditioner & \texttt{DIC} \\ 
& & Tolerance & \texttt{1e-15} \\ 
& & Relative tolerance & 0 \\\hline
\texttt{rho} & \texttt{diagonal} & & \\
\end{tabular}
\end{table}
\end{frame}

\begin{frame}{OpenFOAM Directory Structure}
An OpenFOAM case is divided into:
\begin{itemize}
\item \texttt{0/}: holds initial conditions and boundary conditions for quantities like pressure, temperature, velocity, etc. 
\item \texttt{constant/}: holds thermophysical quantities and some mesh information 
\item \texttt{system/}: contains numerical settings, mesh setup, and simulation settings
\end{itemize}
\end{frame}

\begin{frame}{OpenFOAM \texttt{constant/} Directory}
Contains:
\begin{itemize}
\item \texttt{chemistryProperties}
\item \texttt{combustionProperties}
\item \texttt{dynamicMeshDict}
\item \texttt{reactions}
\item \texttt{thermo.compressibleGas}
\item \texttt{thermophysicalProperties}
\item \texttt{turbulenceProperties}
\end{itemize}
\end{frame}

\begin{frame}{OpenFOAM \texttt{system/} Directory}
Contains:
\begin{itemize}
\item \texttt{blockMeshDict}
\item \texttt{controlDict}
\item \texttt{decomposeParDict}
\item \texttt{setFieldsDict}/\texttt{funkySetFieldsDict}
\item \texttt{fvSchemes}
\item \texttt{fvSolution}
\item files defining post-processing line sampling
\end{itemize}
\end{frame}

\begin{frame}{Simulation Domain Setup}
Besides ignition region, domain is at 1 atm and 300 K
\begin{figure}[t!]
\centering
\includegraphics[width=0.8\textwidth]{../figs/domainBC.png}
%\caption{Geometry and domain setup with boundary conditions}
%\label{fig:domainBC}
\end{figure}%
\end{frame}

\begin{frame}[allowframebreaks]{Parallel Computing}
Domain is decomposed into chunks which are independently processed in parallel, communicating with MPI \cite{walker}. Decomposition defined in \texttt{decomposeParDict}. Several methods for decomposition:
\begin{itemize}
\item \texttt{simple}: define splits in each direction 
\item \texttt{hierarchical}: \texttt{simple}, but with recursive ordering to splits
\item \texttt{scotch}: minimizes boundaries between processors, can set weighting
\item \texttt{manual}: manual cell allocation to each processor 
\end{itemize}
The \texttt{simple} method was used here. 

\begin{figure}[p]
    \centering
    \begin{subfigure}[]{0.4\textwidth}
        \centering
        \includegraphics[width=0.9\textwidth]{../figs/parallel_short.png}
        \caption{Domain decomposed into typical chunks, bad for detonation and AMR load balancing}
        %\label{sfig:shortdecomp}
    \end{subfigure}%
    \begin{subfigure}[]{0.4\textwidth}
        \centering
        \includegraphics[width=0.9\textwidth]{../figs/parallel_long.png}
        \caption{Domain decomposed into long chunks, better for detonation and AMR load balancing}
        %\label{sfig:longdecomp}
    \end{subfigure}
    %\caption{Example domain decomposition techniques for parallel computing}
    %\label{fig:decomp}
\end{figure}%

\end{frame}



\begin{frame}{Adaptive Mesh Refinement}
\end{frame}
