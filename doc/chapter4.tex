\chapter{Summary}
\label{summary}

\section{Project Summary}
In this thesis, the computational fluid dynamics toolbox OpenFOAM was used to simulate detonation waves within a square tube. Different solvers were tested until the hybrid solver \verb|rhoReactingCentralFoam| was decided as best due to the use of the Kurganov and Tadmor central schemes which reproduce compressible, high speed flow with good accuracy. Different ignition methods were tested, and gradient ignition was found to produce results with the least noise. Next, the chemistry was tested for sensitivity and then matched to Chapman-Jouguet targets and the automatic time stepping algorithm was tested to determine how large the CFL number could be allowed to increase before numerical ringing and noise reduced solution accuracy. A value was found that agreed with similar detonation modeling published works. Static computational meshes were then compared to determine the required mesh resolution to resolve shock and detonation structures. The threshold for resolving finer flow field structures as well as the von Neumann spike was determined, as well as the threshold for overall convergence and detonation wave shape integrity. Detonation cells were shown to be able to form with the solver. Adaptive meshing parameters were then explored, namely refinement levels, buffer layers, and refinement/unrefinement bounds. It was shown that refinement levels and the lower refinement bound are exponential in expense and the buffer layers are linear in expense. No more than 4 buffer layers led to a significant increase in solution accuracy, and a normalized pressure gradient lower bound smaller than 0.05 would significantly increase computational cost. Refinement levels started to converge and reproduce the von Neumann spike for 4 and more levels of refinement for the 250-40-1 mesh, which corresponds to a mesh resolution of 4000-640-1 mesh, where the von Neumann spike began to resolve in the static mesh. Lastly, some AMR simulations were compared to static mesh simulations. It was found that using AMR one could potentially reduce the computational expense by 85\% matching a static profile nearly exactly, and reduce computational expense by over 96\% by allowing the detonation spacial position to vary while still targeting peak solution values. 



\section{Next Steps}
\subsection{Areas to Improve}
Increased knowledge as to the effects of AMR in three-dimensional simulations versus the use in two-dimensional simulations may be valuable. Two-dimensional simulations were the primary target of this thesis work due to a balance of feasibility in computational time as well as sufficient expense to justify and show the benefits of AMR. 

An exploration into different tracking parameters other than the normalized gradient of pressure, pressure, temperature, and velocity is useful as a more efficient parameter (i.e., determining a parameter that better describes where refinement is absolutely necessary) may further reduce computational cost. While pressure, temperature, and velocity were all tested during the thesis work, tuning them was found to be more difficult than the normalized pressure gradient. Additionally, combining parameters together may produce smarter AMR ``active'' periods as the refinement is not free and can be potentially more expensive than static mesh cases if not carefully set up. 

Better load balancing in parallel computing in OpenFOAM could use improvement. Currently, intelligent setup and awareness of the detonation itself is required to ensure that the AMR does not offset considerable computational load onto one processor or another as the detonation moves through the domain. For detonation tubes, this is not a problem as long decompositions can be used to balance the processor load. However, for simulations of RDEs or rocket engine combustion chambers that can be inherently unpredictable, smart domain partitioning is difficult. Like adaptive meshing, an adaptive domain decomposition may assist with improving load balancing in these scenarios. 

Further characterization as to how the base mesh affects the solution should be done. While this is seemingly obvious as increased base resolutions will help give a better solution, the AMR can unrefine back to the base resolution, so if large turbulent structures or other larger structures want to be resolved still after adaptive unrefinement, the base resolution needs to be considered. 

The AMR routines in parallel can be unstable. This does not seem to affect the solution, but it is not feasible to ``babysit'' a simulation and restart it if it crashes. The crashes seem to be due to AMR cells not getting communicated correctly with MPI, leading to a disproportionate number of cell faces shared at domain-decomposed boundaries where the MPI communication is occurring. Further work here is needed, and it will likely improve as development continues on the AMR routines. 

\subsection{Future Work}
Further work in detonation modeling in the context of RDEs is the logical next step. This technology is still emerging and having better characterization of the highly chaotic flow field inside the engine will guide better engineering design and analysis towards improved propulsion technology. AMR applied to RDE simulation in OpenFOAM will reduce computational cost, especially for three-dimensional RDE simulations. 

Deflagration to detonation transition modeling is another area to be explored with this solver. Characterizing this will allow the solver to be used in a wider set of scenarios, where sudden detonation of reactants is a potential concern. 

Personally, I think application of AMR simulating the high-speed and chaotic environments of rocket engine combustion chambers in OpenFOAM would be really interesting. Combustion instability within these environments is still very hard to characterize and an ongoing topic of research and development within industry. 

\section{Impact}
The targeted impact area for this research is the field of detonation modeling, with consideration for future use in RDE modeling and other propulsion systems that are expensive computationally. By utilizing the computational tools developed, tested, and validated in this project, existing research in detonation modeling using the computational fluid dynamics toolbox OpenFOAM can be performed more quickly or to a greater extent. Specifically, I showed:
\begin{itemize}
\item The solver \verb|rhoReactingFoam| and \verb|rhoCentralFoam| themselves are unable to model detonations accurately, or at all;
\item The solver \verb|rhoReactingCentralFoam| is able to simulate detonations;
\item How to simulate detonations in OpenFOAM and what the parameters will do to the solution;
\item Detonation cells can be modeled with \texttt{rhoReactingCentralFoam};
\item Results are sensitive to Arrhenius pre-exponential factor order, and care must be taken such that the shock does not decouple from the flame;
\item CFL number needs to be lower than typical high-speed flows due to the reactive nature of detonations;
\item How fine PDE/detonation tube meshes must be to resolve fine detonation and shock structure;
\item Certain AMR parameters have different effects computationally, and tuning certain parameters over others will optimize the overall cell count by placing refinement cells where they are truly needed;
\item Adaptive mesh refinement can decrease computation cost for detonation modeling up to 96\% while remaining true to the static mesh resolution results. 
\end{itemize}

