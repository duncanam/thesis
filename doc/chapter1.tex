\chapter{Introduction}
\label{introchap}

\section{Motivation}
Some modern combustion problems such as performance of internal combustion engines has been well-characterized for years. This has allowed for improved engine performance, efficiency, and reliability. Internal combustion engines are just a single area within the field of study of reacting flows, which can include topics such as wildfires, supernovae, explosions, ocean biomechanics, and aircraft or rocket propulsion. Considering aircraft and rocket propulsion, modeling efforts of these systems can be increasingly difficult due to the reactive and potentially high-Mach nature of their flows. Computational fluid dynamics (CFD) is often used to model these types of reacting flows, in which the domain is discretized into smaller chunks typically in which flow equations such as Navier-Stokes is solved. With high-speed fluid flows in CFD, the computational grid must be very fine in order to remain accurate. The addition of chemistry solving adds further complexity and time required for a solution. When combined, a high-Mach reacting flow is very computationally expensive and exceedingly difficult to model accurately. 

An area of interest within aircraft and rocket propulsion are detonation engines. Two main subtypes of these engines exist: the pulse detonation engine (PDE) and the rotating detonation engine (RDE). Both of these propulsion technologies rely on the ignition of a detonation wave, a supersonic flame front, to provide the means of propulsion. The detonation itself is usually quite thin and is accompanied by high pressure and temperature. With such a fast and physically small flow feature, detonations require very fine grids in order to be accurately captured. 


