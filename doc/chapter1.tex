\chapter{Introduction}
\label{introchap}

\section{Motivation}
Some modern combustion problems such as performance of internal combustion engines has been well-characterized for years. This has allowed for improved engine performance, efficiency, and reliability. Internal combustion engines are just a single area within the field of study of reacting flows, which can include topics such as wildfires, supernovae, explosions, ocean biomechanics, and aircraft or rocket propulsion. Considering aircraft and rocket propulsion, modeling efforts of these systems can be increasingly difficult due to the reactive and potentially high-Mach nature of their flows. Computational fluid dynamics (CFD) is often used to model these types of reacting flows, in which the domain is discretized into smaller chunks typically in which flow equations such as Navier-Stokes is solved. With high-speed fluid flows in CFD, the computational grid must be very fine in order to remain accurate. The addition of chemistry solving adds further complexity and time required for a solution. When combined, a high-Mach reacting flow is very computationally expensive and exceedingly difficult to model accurately. 

An area of interest within aircraft and rocket propulsion are detonation engines. Two main subtypes of these engines exist: the pulse detonation engine (PDE) and the rotating detonation engine (RDE). PDEs and RDEs are interesting as they can theoretically have a higher efficiency than traditional propulsive device such as a turbofan or some rocket engines. Both of these propulsion technologies rely on the ignition of a detonation wave, a supersonic flame front, to provide the means of propulsion. The detonation itself is usually quite thin and is accompanied by high pressure and temperature. With such a fast and physically small flow feature, detonations require very fine grids in order to be accurately captured. This adds to the computational expense. A solution to this problem is to adaptively refine the grid (mesh), allowing for a finer mesh in locations where it is more critical such as around a detonation wave, and less fine of a mesh in areas of the domain that do not have complex flow features. By doing this, the computer is only solving grid points where necessary, so computational resources aren't wasted. This is called adaptive mesh refinement (AMR). Since PDEs and RDEs have strong shockwaves within their domains, they are an ideal candidate for AMR. With AMR, PDE and RDE computational research can progress more efficiently as accurate flow-field solutions can be obtained with less computational expense. 

\section{Objectives and Scope}
The purpose of this thesis work is to solidify AMR and modeling techniques using the open-source computational fluid dyanamics (CFD) toolbox known as OpenFOAM within the context of detonations for PDEs and RDEs. OpenFOAM has several benefits over an in-house CFD code:

\begin{itemize}
    \item better geometric flexibility for input 
    \item input structure is nearly identical for most solvers, allowing for reduced modeling time when comparing solvers for similar cases
    \item documentation is more widely-available
\end{itemize}

\noindent Some of the cases that will be explored here is the shock-capturing accuracy of the selected solver as well as simulation of one-dimensional detonations, two dimensional detonations, and three dimensional detonations.

Another objective of this thesis work is to take the detonation modeling techniques with AMR in OpenFOAM such that the University of Colorado's Turbulence and Energy Systems Laboratory (TESLa) can use them for more in-depth and important research, such as RDE research. As such, the focuses during this thesis is centered more around solver methods, exploration of features, tweaking parameters, and building up a framework to start on for further research instead of in-depth solver building with mathematical foundations or heavy focus on turbulence and combustion mathematical models. 

\section{Thesis Organization}
