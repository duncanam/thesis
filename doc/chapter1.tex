\chapter{Introduction}
\label{introchap}


\section{Motivation}
Many challenges associated with modern combustion systems, such as improving the efficiency and performance of internal combustion engines, have been the focus of considerable research for the past several decades. This focus has resulted in improved engine performance, efficiency, and reliability. However, internal combustion engines are just one area within the much broader field of study of reacting flows, which can include topics such as diverse as wildfires, supernovae, explosions, ocean biogeochemistry, and aircraft or rocket propulsion.

Considering aircraft and rocket propulsion, in particular, modeling of these systems is made exceedingly difficult by the reactive and potentially high-Mach nature of the turbulent flows involved in most such applications. Computational fluid dynamics (CFD) is often used to model these flows, where the system domain is discretized using a computational grid, on which flow equations such as the Navier-Stokes equations are solved. For CFD of high-speed turbulent reacting flows, however, this grid must be very fine in order to maintain accurate solutions of the governing equations. The addition of chemical reactions further complicates the calculation and can substantially increase the time required for a solution. As a result, CFD of high-speed turbulent reacting flows is computationally expensive and the use of cost-reducing methods, such as subgrid-scale modeling for large eddy simulations (LES), can substantially reduce physical accuracy.  

These challenges are all particularly relevant to a growing area of interest within aircraft and rocket propulsion; namely, the use of detonation engines to achieve high efficiency and high speed propulsion. Two main subtypes of these engines exist: the pulsed detonation engine (PDE) and the rotating detonation engine (RDE). Both PDEs and RDEs are attractive since, in principle, they can theoretically produce higher efficiencies than traditional propulsive devices such as turbofans and some rocket engines. Both of these propulsion technologies rely on the ignition of a detonation wave (i.e., a supersonic flame front) to provide the means of propulsion. The detonation wave itself is usually quite thin and is accompanied by high pressures and temperatures. 

With such fast and physically small flow features, detonations require very fine grids in order to be accurately captured. This adds to the computational expense of turbulent reacting flow simulations relevant to PDEs and RDEs. A potential solution to this problem is to adaptively refine the grid (mesh), allowing for a finer meshes in locations where high resolutions are more critical, such as around a detonation wave, and a less refined mesh in areas of the domain that do not have complex flow features. By doing this, the computer is only solving grid points where necessary, so computational resources are deployed as efficiently as possible. The resulting procedure is called adaptive mesh refinement (AMR). 

Because PDEs and RDEs have strong detonation waves within their domains, they are ideal candidates for the efficiency and performance gains enabled by AMR. With AMR, PDE and RDE computational research can progress more efficiently, since accurate flow-field solutions can be obtained with less computational expense. The present MS thesis is focused, in particular, on developing an AMR capability for the simulation of detonation waves found in PDEs and RDEs.

\section{Objectives and Scope}
The primary objective of this thesis is to implement and test AMR techniques for simulations of detonation waves found in PDEs and RDEs using the open-source computational fluid dynamics toolbox known as OpenFOAM \cite{weller}. Compared to an in-house CFD code use previously for detonation engine research \cite{towery1}, OpenFOAM has several benefits, including increased geometric flexibility, an input structure that is nearly identical for most solvers, allowing for reduced modeling time when comparing solvers for similar cases, and more widely available documentation is more widely-available.

The cases studied in this thesis demonstrate the shock-capturing accuracy of the solver, as well as the simulation of one-, two-, and three-dimensional detonations. Another objective of this thesis work is to build on previous work \cite{towery1} and develop the detonation modeling techniques with AMR in OpenFOAM such that the Turbulence and Energy Systems Laboratory (TESLa) at the University of Colorado, Boulder can use them for more in-depth and important research, such as the optimization of geometrically-complex RDE systems. As such, the focus of this thesis is centered on solver methods, exploration of solver features, AMR configuration sensitivites, and building a framework that can serve as the basis for further research. 

\section{Previous Detonation Modeling Work}
In order to be able to do the research outlined in this document, many resources were utilized and many others have layed the path in which this work was built off from. 

The primary work this research was build off of was done by Towery in ``Examination of Turbulent Flow Effects in Rotating Detonation Engines'' \cite{towery1}. The work done in this paper examined both PDE or detonation tube computational modeling as well as RDE modeling. The numerical code used was developed in-house by TESLa, written in FORTRAN and was massively scalable in parallel. However, this code lacked the geometric and generic flexibility of OpenFOAM, hence the research done in this document. Towery's paper, published in 2014, examined numerical methods for large eddy simulation (LES) of PDEs and RDEs in two and three dimensions. Reynolds averaged turbulence models were also compared to the higher resolution LES simulation results. Both single and two-step reactions kinetics were used in the simulation of detonations in comparison to the single step kinetics used in this document. AMR was not utilized for simulation. 

In 2013 Schwer and Kailasanath published ``Fluid dynamics of rotating detonation engines with hydrogen and hydrocarbon fuels''\cite{schwer1}. This paper primarily discussed the effects of different hydrocarbon fuels on RDEs as well as as their numerical methods for simulating RDEs and detonation waves in tubes. Their tube detonation simulations were performed using a flux-corrrected-transport (FCT) algorithm scaled across hundreds of cores in parallel. Their work showed that RDEs are not affected by thermodynamic differences in hydrocarbon fuels in comparison to hydrogen fuel and that ideal detonation cycle theory could be used for baseline performance. In order to reach conclusions about RDE performance, studies were done on PDE and detonation tube modeling to verify the validity of their solution methods, much like the work done in this document. Detonation tube numerical results were compared to CJ theory and detonation cellular structure was also produced. 

Marcantoni, Tamagno, and Elaskar in 2017 published ``rhoCentralRfFoam: An OpenFOAM solver for high speed chemically active flows - Simulation of planar detonations -''\cite{marcantoni}. Very similar to the work done in this document but without AMR capability, with \verb|rhoCentralRfFoam| they took the Kurganov central schemes that model compressible flow so well from \verb|rhoCentralFoam| and added reaction modeling capability. Various mass fraction compositions were tested in air as well as chemistry validity spanning into multi-reaction chemical models. They selected a maximum Courant number of 0.25 for the one-dimensional planar detonation tests. \verb|rhoCentralRfFoam| was found to agree with CJ theory as well as other numerical codes such as CEA \cite{CEA}. 

Dinh, Yoshida, and Ishikura published ``Simulation of Rotating Detonation Engine by OpenFOAM''\cite{dinh} in 2019. They developed an in-house proprietary OpenFOAM code for modeling detonations. This code had the ability to use AMR and was run on a single processor. The AMR parameters used for tracking refinement were normalized reaction heat as well as normalized velocity gradient at an interval of every 10 computational time steps. Together, both thresholds had to be met in order for AMR to occur. Comparisons with analytical theory such as CJ theory or experimental results were not performed in this work, and research done was more exploratory to determine the applicability of the solver. 

Additionally in 2019, Kim and Kim published ``Numerical method to simulate detonative combustion of hydrogen-air in a containment''\cite{kim}. This research utilized an OpenFOAM solver with central-upwind schemes to study hydrogen-air detonation within confined spaces. Arrhenius rate modeling with seven-step chemistry component reactions were utilized. Similar numerical schemes for solving the reactive Euler equations were used in this work compared to the solvers in this document. Shock tube comparison was done for solver validation as well as further comparison to confined detonation cases, and a CFL number of 0.1 was prescribed. It was found that their solver was capable of capturing shock structures and fluid modeling to match theory as well being able to replicate accurate confined hydrogen-air detonations. 

\section{Thesis Organization}
This document is broken up into several chapters. The next chapter discusses the fundamental governing equations of the flows studied here. It also introduces various numerical solution techniques, as well as how they are implemented in order to approximate a solution to the governing equations. This chapter also discusses OpenFOAM as a whole, as well as specific details as to how detonations were modeled with the toolbox. Some initial failures are shown here and solutions or alternatives used to overcome the failures. The following chapter compares different solver parameters used to model detonations and how the variation of the parameters affects the solution. Different levels of accuracy are discussed as well as well as generic results and comparisons overall. Conclusions and directions for future work are provided at the end.